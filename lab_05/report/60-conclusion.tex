\chapter*{Заключение}
\addcontentsline{toc}{chapter}{Заключение}

В данной лабораторной работе были рассмотрены
основополагающие материалы которые в дальнейшем потребовались
при реализации конвейера.
Были рассмотрены схемы (рис. \ref{ref:d0} - \ref{ref:d1}) показывающие алгоритм конвейера.
Также были разобраны листинги рис 3.1-3.2,
показывающие работу контейнера и был приведен рис. \ref{ref:res},
показывающий корректную работу конвейера.
Был произведен сравнительный анализ рис. \ref{ref:time}.

В рамках выполнения работы решены следующие задачи:

\begin{enumerate}
	\item изучили освоили конвейерную обработку данных;
	\item применили изученные основы для реализации конвейерной обработки данных;
	\item получили практические навыки;
	\item получили статистику выполнения программы;
	\item описали и обосновали полученные результаты;
	\item выбрали и обосновали языка программирования, для решения данной задачи.
\end{enumerate}
