\newpage
\chapter*{Введение}
\addcontentsline{toc}{chapter}{Введение}
Параллельные вычисления часто используются для увеличения скорости выполнения
программ. Однако приемы, применяемые для однопоточных машин, для
параллельных могут не подходить. Конвейерная обработка данных является
популярным приемом при работе с параллельными машинами.

% В данной лабораторной работе будет рассмотрен и реализован метода конвейерных вычислений.

Целью данной работы является изучение и реализация метода конвейерных вычислений.

В рамках выполнения работы необходимо решить следующие задачи.

\begin{enumerate}
	\item Изучения основ конвейерной обработки данных;
	\item Применение изученных основ для реализации конвейерной обработки данных;
	\item Получения практических навыков;
	\item Получение статистики выполнения программы;
	\item Описание и обоснование полученных результатов;
	\item Выбор и обоснование языка программирования, для решения данной задачи.
\end{enumerate}