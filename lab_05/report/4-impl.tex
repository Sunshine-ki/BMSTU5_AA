\chapter{Технологическая часть}

\section{Выбор ЯП}

В данной лабораторной работе использовался язык программирования - C\# \cite{Microsoft}.
Данный язык был выбран, потому что он является нативным.
В качестве среды разработки я использовала Visual Studio Code \cite{Vs}.
Visual Studio Code подходит не только для  Windows \cite{Win},
но и для Linux \cite{Lin}, это причина,
по которой я выбрала VS code,
т.к. у меня установлена ОС Ubuntu 18.04.4 \cite{Ubuntu}.
В моей архитектуре присутствует 8 ядер.

% Время работы алгоритмов было замерено с помощью класса Stopwatch. Многопоточное программирование было
% реализовано с помощью пространства имен System.Threading.

\section{Сведения о модулях программы}

Данная программа состоит из файла Program.cs, который содержит код программы.

На листингах 3.1-3.2 представлен основной код программы.

\begin{lstlisting}[label=some-code,caption=Метод создания и запуска потоков]
public static void MainTread(Queue<intPtr> queue)
{
	Console.WriteLine("Process:\n");
	
	ThreadArgs args = new ThreadArgs(queue);
	
	Thread firstThread = new Thread(new ParameterizedThreadStart(Conveyor));
	Thread secondThread = new Thread(new ParameterizedThreadStart(Conveyor));
	Thread thirdThread = new Thread(new ParameterizedThreadStart(Conveyor));
	
	firstThread.Start(args);
	secondThread.Start(args);
	thirdThread.Start(args);

	while (args.firstQueue.Count != 0)
	{
		if (!firstThread.IsAlive)
		{
			firstThread = new Thread(new ParameterizedThreadStart(Conveyor));
			firstThread.Start(args);
		}
	
		if (!secondThread.IsAlive)
		{
			secondThread = new Thread(new ParameterizedThreadStart(Conveyor));
			secondThread.Start(args);
		}
	
		if (!thirdThread.IsAlive)
		{
			thirdThread = new Thread(new ParameterizedThreadStart(Conveyor));
			thirdThread.Start(args);
		}
	}
	
	firstThread.Join(); secondThread.Join(); thirdThread.Join();
}
\end{lstlisting}

\begin{lstlisting}[label=some-code,caption=Конвейер]
public static void Conveyor(object obj)
{
	ThreadArgs args = (ThreadArgs)obj;
	int max, min, count;
	intPtr array;

	lock (args.firstQueue)
	{
		// Get array from the first queue.
		array = args.firstQueue.Dequeue();
	}

	lock (firstStage)
	{
		// The first tape is running.
		max = FindMax(array);
	}

	lock (args.secondQueue)
	{
		// Added element to the second queue.
		args.secondQueue.Enqueue(array);
	}

	lock (secondStage)
	{
		// The second tape is running.
		lock (args.secondQueue)
		{
			// Get array from the second queue.
			array = args.secondQueue.Dequeue();
		}
		min = FindMin(array);
	}

	lock (args.thirdQueue)
	{
		args.thirdQueue.Enqueue(array);
	}

	lock (thirdStage)
	{
		lock (args.thirdQueue)
		{
			array = args.thirdQueue.Dequeue();
		}
		count = FindCount(array, (max - min) / 2);
	}
}
\end{lstlisting}



\section{Вывод}

В данном разделе были разобраны листинги рис 3.1-3.2, показывающие работу конвейера.