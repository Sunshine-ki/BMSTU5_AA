\chapter{ Технологический раздел}
\label{cha:design}

\section{Выбор ЯП}

В данной лабораторной работе использовался язык программирования - python \cite{bib1}.
Данный язык простой и понятный, также я знакома с ним.
Поэтому данный язык был выбран. 
В качестве среды разработки я использовала Visual Studio Code \cite{bib2}, т.к. считаю его достаточно удобным и легким.
Visual Studio Code подходит не только для  Windows \cite{bib3}, но и для Linux \cite{bib4}, это еще одна причина, по которой я выбрала VS code, т.к. у меня установлена ОС Ubuntu 18.04.4 \cite{bib5}.

\section{Требования к программному обеспечению}

Входными данными являются две матрицы A и B.
Количество столбцов матрицы A долджно быть равно количеству строк матрицы B. 

На выходе получается результат умножения, введенных пользователем, матриц.

\section{Сведения о модулях программы}

Данная программа разбита на модули.

\begin{itemize}
    \item main.py - файл, содержащий точку входа в программу. В нем происходит общение с пользователем и вызов алгоритмов.
    \item matrix.py - файл, содержащий класс matrix.
    \item matrix\_multiplication.py - файл, содержащий реализации алгоритмов умножения матриц.
\end{itemize}

На листингах 3.1-3.4 представлен код программы.

\begin{lstlisting}[label=some-code,caption=Главная функция main]
def main():
    output("МАТРИЦА A", BLUE)
    n = int(input(GREEN + "Введите кол-во строк: "))
    m = int(input(GREEN + "Введите кол-во столбцов: "))
    output("Введите матрицу:", GREEN)
    matrixA = Matrix(n, m, [[int(j) for j in input(GREEN).split()]
                            for i in range(n)])

    output("МАТРИЦА B", BLUE)
    k = int(input(GREEN + "Введите кол-во строк: "))
    l = int(input(GREEN + "Введите кол-во столбцов: "))
    output("Введите матрицу:", GREEN)
    matrixB = Matrix(k, l, [[int(j) for j in input(GREEN).split()]
                            for i in range(k)])

    if m != k:
        output("Некорректные размеры матриц!", RED)
        return

    matrixC = multiplication(matrixA, matrixB)
    matrixC.output()
\end{lstlisting}

\begin{lstlisting}[label=some-code,caption=Класс matrix]
class Matrix:
    n, m = 0, 0
    matrix = list()

    def __init__(self, n, m, list_of_lists=None):
        self.n, self.m = n, m
        if list_of_lists:
            self.matrix = deepcopy(list_of_lists)
        else:
            self.matrix = np.full((n, m), 0)

    def output(self):
        print(TURQUOISE, end='')
        for i in range(self.n):
            for j in range(self.m):
                print(self.matrix[i][j], end=' ')
            print()

    def fill(self, list_of_lists):
        self.matrix = deepcopy(list_of_lists)

    def size(self):
        return (self.n, self.m)

    def __getitem__(self, index):
        return self.matrix[index]
\end{lstlisting}

\begin{lstlisting}[label=some-code,caption=Стандарный алгоритм умножения матриц]
    def multiplication(matrixA, matrixB):
        n = matrixA.n
        m = matrixB.m
        q = matrixB.n

        matrixC = Matrix(n, m)

        for i in range(n):
            for j in range(m):
                for k in range(q):
                    matrixC[i][j] += (matrixA[i][k] * matrixB[k][j])

        return matrixC
\end{lstlisting}

\begin{lstlisting}[label=some-code,caption=Алгоритм Винограда]
    def WinogradMult(matrixA, matrixB):
        n = matrixA.n
        m = matrixB.m
        q = matrixB.n

        matrixC = Matrix(n, m)

        tempA = np.full(n, 0)
        for i in range(n):
            for j in range(1, q, 2):
                tempA[i] += matrixA[i][j - 1] * matrixA[i][j]

        tempB = np.full(m, 0)
        for i in range(m):
            for j in range(1, q, 2):
                tempB[i] += matrixB[j - 1][i] * matrixB[j][i]

        for i in range(n):
            for j in range(m):
                matrixC[i][j] -= (tempA[i] + tempB[j])
                for k in range(1, q, 2):
                    matrixC[i][j] += (matrixA[i][k-1] + matrixB[k][j]) * \
                        (matrixA[i][k] + matrixB[k-1][j])
        if q % 2:
            for i in range(n):
                for j in range(m):
                    matrixC[i][j] += matrixA[i][q-1] * matrixB[q-1][j]

        return matrixC
\end{lstlisting}

\section{Тестирование}

В данном разделе будет приведена таблица с тестами (таблица \ref{table:ref1}).
% \ref{table:ref1}, в которой четко отражено тестирование программы

\begin{table}[ht]
    \centering
    \caption{Таблица тестов}
    \label{table:ref1}
    \begin{tabular}{ | l | l | l |}
        \hline
        Матрица A       & Матрица B           & Результат                               \\ \hline
        2 2 1 0 0 1     & 2 2 1 0 0 1         & Ответ верный                            \\ \hline
        3 2 2 3 1 0 2 2 & 2 4 2 2 1 9 4 2 8 1 & Ответ верный                            \\ \hline
        2 1 2 1         & 12 12               & Ответ верный                            \\ \hline
        2 2 1 0 0 1     & 1 1 0               & Сообщение о неверном вводе размерностей \\ \hline
        0 0             & 0 0                 &                                         \\ \hline
        \hline
    \end{tabular}
\end{table}

Все тесты пройдены.

\section{Вывод}

В данном разделе был разобран выбор языка программирования, а также среды разработки.
Разобраны требования к программному обеспечению.
Протестированная программа. 
