\chapter{ Технологический раздел}
\label{cha:design}

\section{Выбор ЯП}

В данной лабораторной работе использовался язык программирования - python \cite{bib1}.
Я знакома с данным языком.
Поэтому он был выбран. 
В качестве среды разработки я использовала Visual Studio Code \cite{bib2}, т.к. считаю его достаточно удобным и легким.
Visual Studio Code подходит не только для  Windows \cite{bib3}, но и для Linux \cite{bib4}, это еще одна причина, по которой я выбрала VS code, т.к. у меня установлена ОС Ubuntu 18.04.4 \cite{bib5}.

\section{Требования к программному обеспечению}

Входными данными являются размерность массива n и n элементов массива.

На выходе получается отсортированный массив.

\section{Сведения о модулях программы}

Данная программа разбита на модули:

\begin{itemize}
	\item main.py - Файл, содержащий точку входа в программу. В нем происходит общение с пользователем и вызов алгоритмов;
	\item sort.py - Файл, содержащий алгоритмы сортировок.
	\item arr.py - файл, содержащие методы для работы с массивом.
\end{itemize}

На листингах 3.1-3.5 представлен код программы.

\begin{lstlisting}[label=some-code,caption=Главная функция main]
def main():
	arr = inputArr()

	output("Введенный массив:", GREEN)
	outputArr(arr)

	arr_bubble = sort_bubble(arr)

	output("Отсортированный массив (Пузырек):", GREEN)
	outputArr(arr_bubble)

	arr_insert = sort_insert(arr)
	output("Отсортированный массив (Вставками):", GREEN)
	outputArr(arr_insert)

	arr_quick = quick_sort(arr)
	output("Отсортированный массив (Quick sort):", GREEN)
	outputArr(arr_quick)
\end{lstlisting}

\begin{lstlisting}[label=some-code,caption=Сортировка пузырьком]
	def sort_bubble(arr):
    n = len(arr) - 1

    for i in range(n):
        for j in range(n - i):
            if arr[j] > arr[j + 1]:
                arr[j], arr[j+1] = arr[j+1], arr[j] 

    return arr
\end{lstlisting}

\begin{lstlisting}[label=some-code,caption=Сортировка вставками]
def sort_insert(arr):
    n = len(arr)

    for i in range(1, n):
        temp = arr[i]
        j = i
        while j > 0 and arr[j - 1] > temp:
            arr[j] = arr[j - 1]
            j -= 1
        arr[j] = temp

    return arr
\end{lstlisting}

\begin{lstlisting}[label=some-code,caption=QuickSort]
def sort_quick_recursion(arr, begin, end):
    if begin < end:
        # Опорная точка. pivot.
        p = begin
        for i in range(begin + 1, end):
            if arr[i] < arr[begin]:
                p += 1
                arr[p], arr[i] = arr[i], arr[p]
        arr[begin], arr[p] = arr[p], arr[begin] 
        
        sort_quick_recursion(arr,  begin, p)
        sort_quick_recursion(arr, p + 1, end)

	return arr
	
def quick_sort(arr):
    return sort_quick_recursion(arr, 0, len(arr) - 1)
\end{lstlisting}


\begin{lstlisting}[label=some-code,caption=Методы для работы с массивом]
def inputArr():
    n = int(input("Введите n: "))
    arr = list()

    if (not n):
        return
    for _ in range(n):
        arr.append(int(input("Введите элемент: ")))

    return arr


def outputArr(arr):
    for i in range(len(arr)):
        print(arr[i], end=" ")
    print()
\end{lstlisting}


\section{Тестирование}


В данном разделе будет приведена таблица с тестами (таблица \ref{table:ref1}).

\begin{table}[ht]
	\centering
	\caption{Таблица тестов}
	\label{table:ref1}
	\begin{tabular}{ | l | l | l |}
		\hline
		Массив    & Результат & Результат    \\ \hline
		5 4 3 2 1 & 1 2 3 4 5 & Ответ верный \\ \hline
		1 2 3 4 5 & 1 2 3 4 5 & Ответ верный \\ \hline
		1 -1 2 0  & -1 0 1 2  & Ответ верный \\ \hline
		1 -1 2 0  & -1 0 1 2  & Ответ верный \\ \hline
		-1 -2 -3  & -3 -2 -1  & Ответ верный \\ \hline
		1 1 1 1   & 1 1 1 1   & Ответ верный \\ \hline
		0         &           & Ответ верный \\ \hline
		\hline
	\end{tabular}
\end{table}

Все тесты пройдены.

\section{Вывод}

В данном разделе был выбран и обоснован язык программирования.
Были разобраны листинги 3.1-3.5 с кодом программы.
А также приведены тесты (таблица \ref{table:ref1}).