\Conclusion % заключение к отчёту

В данной лабораторной работе были рассмотрены
основополагающие материалы которые в дальнейшем потребовались
при параллельной и однопоточной реализации алгоритма трассировки лучей.
Были рассмотрены схемы однопоточной (рис. \ref{ref:d3}) и многопоточной
(рис. \ref{ref:d0} - \ref{ref:d2}) реализации ранее разобранного алгоритма.
Также были разобраны листинги рис 3.1-3.4, показывающие работу как однопоточного, так и многопоточного
алгоритма трассировки лучей и был приведен рис. \ref{ref:res}, показывающий корректную работу алгоритма.
Было произведено и показано сравнение работы алгоритма на разных количествах потоков (рис. \ref{ref:time}).

В рамках выполнения работы решены следующие задачи.

\begin{enumerate}
	\item Изучены основы параллельных вычислений.
	\item Применены изученные основы при реализации многопоточности на материале трассировки лучей.
	\item Получены практические навыки.
	\item Произведен сравнительный анализ параллельной и однопоточной реализации алгоритма трассировки лучей.
	\item Экспериментально подтверждены различия во временной эффективность реализации однопоточной и многопоточной трассировки лучей.
	\item Описаны и обоснованы полученные результаты.
	\item Выбран и обоснован языка программирования, для решения данной задачи.
\end{enumerate}
