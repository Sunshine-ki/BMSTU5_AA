%% Преамбула TeX-файла

% 1. Стиль и язык
\documentclass[utf8x, 12pt]{G7-32} % Стиль (по умолчанию будет 14pt)
% report
% Остальные стандартные настройки убраны в preamble.inc.tex.
\include{preamble.inc}

% Настройки листингов.
\ifPDFTeX
\include{listings.inc}
\else
\usepackage{local-minted}
\fi

% Полезные макросы листингов.
\include{macros.inc}

\begin{document}

\frontmatter 

\include{00-abstract}

\tableofcontents

\newpage
\chapter*{Введение}
\addcontentsline{toc}{chapter}{Введение}
Параллельные вычисления часто используются для увеличения скорости выполнения
программ. Однако приемы, применяемые для однопоточных машин, для
параллельных могут не подходить. Конвейерная обработка данных является
популярным приемом при работе с параллельными машинами.

% В данной лабораторной работе будет рассмотрен и реализован метода конвейерных вычислений.

Целью данной работы является изучение и реализация метода конвейерных вычислений.

В рамках выполнения работы необходимо решить следующие задачи.

\begin{enumerate}
	\item Изучения основ конвейерной обработки данных;
	\item Применение изученных основ для реализации конвейерной обработки данных;
	\item Получения практических навыков;
	\item Получение статистики выполнения программы;
	\item Описание и обоснование полученных результатов;
	\item Выбор и обоснование языка программирования, для решения данной задачи.
\end{enumerate}

\mainmatter % это включает нумерацию глав и секций в документе ниже

\include{2-analysis}
\chapter{ Констукторский раздел}
\label{cha:design}

В данном разделе мы рассмотрим схемы.

На рис. \ref{ref:d0} показана схема алгоритма главного потока.

\begin{figure}[ht!]
	\centering{
		\includegraphics[width=0.8\textwidth]{img/d4.png}
		\caption{Схема алгоритма главного потока.}
		\label{ref:d0}}
\end{figure}

На рис. \ref{ref:d1} показана схема работы потока, который разбивает
экран вертикально. 

\begin{figure}[ht!]
	\centering{
		\includegraphics[width=0.8\textwidth]{img/d1.png}
		\caption{Вертикальное разбиение экрана.}
		\label{ref:d1}}
\end{figure}

На рис. \ref{ref:d2} показана схема работы потока, который разбивает
экран горизонтально.

\begin{figure}[ht!]
	\centering{
		\includegraphics[width=0.8\textwidth]{img/d2.png}
		\caption{Горизонтальное разбиение экрана.}
		\label{ref:d2}}
\end{figure}

На рис. \ref{ref:d3} показана схема однопоточного алгоритма трассировки
лучей.

\begin{figure}[ht!]
	\centering{
		\includegraphics[width=0.3\textwidth]{img/d3.png}
		\caption{Однопоточная трассировка лучей.}
		\label{ref:d3}}
\end{figure}



\section{Вывод}

В данном разделе были рассмотрены схемы однопоточной (рис. \ref{ref:d3}) и многопоточной
(рис. \ref{ref:d0} - \ref{ref:d2}) реализации алгоритма обратной трассировки лучей







\chapter{ Технологический раздел}
\label{cha:design}

\section{Выбор ЯП}

В данной лабораторной работе использовался язык программирования - C\# \cite{bib1}.
Данный язык является нативным, в отличие от языка программирования python. Также я знакома с ним.
Поэтому данный язык был выбран. 
В качестве среды разработки я использовала Visual Studio Code \cite{bib2}, т.к. считаю его достаточно удобным и легким.
Visual Studio Code подходит не только для  Windows \cite{bib3}, но и для Linux \cite{bib4}, это еще одна причина, по которой я выбрала VS code, т.к. у меня установлена ОС Ubuntu 18.04.4 \cite{bib5}.
В моей архитектуре присутствует 8 ядер.

\section{Сведения о модулях программы}

Данная программа разбита на модули:

\begin{itemize}
	\item Program.cs - Файл, содержащий точку входа в программу.
	\item MainFrom.cs - Файл, содержащий основной код программы.
\end{itemize}

На листингах 3.1-3.4 представлен основной код программы.

\begin{lstlisting}[label=some-code,caption=Метод создания и запуска потоков]
private void DrawScene()
{
	List<Thread> listThread = new List<Thread>();

	int step = 50;
	for (int i = 0; i < (int)SizeObjects.WidthCanvas; i += step)
	{
		listThread.Add(new Thread(new ParameterizedThreadStart(FuncHorizontally)));
		listThread[listThread.Count - 1].Start(new Limit(i, i + step));
	}

	foreach (var elem in listThread)
		elem.Join();

	_imgBox.Image = _img;
}
\end{lstlisting}

\begin{lstlisting}[label=some-code,caption=Метод потока разбиения горизонтально]
public void FuncHorizontally(object obj)
{
	Limit limit = (Limit)obj;
	for (Int32 i = limit.begin; i < limit.end; i++)
		for (Int32 j = 0; j < (int)SizeObjects.HeightCanvas; j++)
			_img.SetPixel(i, j, TraceRay(new Point(i, j)));
}
\end{lstlisting}

\begin{lstlisting}[label=some-code,caption=Метод потока разбиения вертикально]
public void FuncVertically(object obj)
{
	Limit limit = (Limit)obj;
	for (Int32 i = 0; i < (int)SizeObjects.WidthCanvas; i++)
		for (Int32 j = limit.begin; j < limit.end; j++)
			_img.SetPixel(i, j, TraceRay(new Point(i, j)));
}
\end{lstlisting}

\begin{lstlisting}[label=some-code,caption=Однопоточный метод трассировки лучей.]
private void funcTrace()
{
	for (Int32 i = 0; i < (int)SizeObjects.WidthCanvas; i++)
		for (Int32 j = 0; j < (int)SizeObjects.HeightCanvas; j++)
			_img.SetPixel(i, j, TraceRay(new Point(i, j)));
}
\end{lstlisting}

\section{Тестирование}

В данном разделе приведен рис. \ref{ref:res} на котором показан результат работы программы.

\begin{figure}[ht!]
	\centering{
		\includegraphics[width=0.8\textwidth]{img/res.png}
		\caption{Результат работы программы.}
		\label{ref:res}}
\end{figure}


\section{Вывод}

В данном разделе были разобраны листинги рис 3.1-3.4, показывающие работу как однопоточного, так и многопоточного
алгоритма трассировки лучей. Также приведен рис. \ref{ref:res}, показывающий корректную работу алгоритма.
\chapter{Экспериментальная часть}

В данном разделе будет произведено сравнение вышеизложенных алгоритмов.

\section{Временные характеристики}


Для сравнения возьмем квадратные матрицы размерностью [10, 20, 30,\dots,100]. 
Так как подсчет умножения матриц считается короткой задачей, воспользуемся усреднением массового эксперимента. 
Для этого сложим результат работы алгоритма n раз (n >= 10), после чего поделим на n. 
Тем самым получим достаточно точные характеристики времени. 
Сравнение произведем при n = 50.
Результат можно увидеть на рис. \ref{fg:ref3}. 

\begin{figure}[ht!]
	\centering{
		\includegraphics[width=0.8\textwidth]{img/time1.png}
		\caption{Временные характеристики на четных размерах матриц}
		\label{fg:ref3}}
\end{figure}

На рис. \ref{fg:ref4} показана работа алгоритмов с матрицами, размерностью [11, 21, 31,\dots,91].

\begin{figure}[ht!]
	\centering{
		\includegraphics[width=0.8\textwidth]{img/time2.png}
		\caption{Временные характеристики на нечетных размерах матриц}
		\label{fg:ref4}}
\end{figure}

\section{Сравнительный анализ алгоритмов}

\section{Вывод}

В данном разделе было произведено сравнение количества затраченного вре­мени вышеизложенных алгоритмов.
Самым быстрым оказался модифицированный алгоритм Винограда.


\backmatter %% Здесь заканчивается нумерованная часть документа и начинаются ссылки и
%% заключение

\Conclusion % заключение к отчёту

Алгоритмы умножения матриц являют­ся очень важными алгоритмами
В этой лабораторной работе были изучены алгоритмы ... (тут ссылки на все, что было сделано)

В рамках выполнения работы решены следующие задачи.

\begin{enumerate}
	\item Изучен и реализован на выбранном ЯП стандартный алгоритм умножения матриц.
	\item Изучен и реализован алгоритм Винограда умножения матриц.
	\item Оптимизирован алгоритм Винограда умножения матриц.
	\item Произведены сравнения временные характеристик вышеизложенных алгоритмов.
	\item Оценены алгоритмы.
\end{enumerate}

% \include{61-biblio}

\begin{thebibliography}{1}
	\def\selectlanguageifdefined#1{
		\expandafter\ifx\csname date#1\endcsname\relax
		\else\language\csname l@#1\endcsname\fi}
	\ifx\undefined\url\def\url#1{{\small #1}}\else\fi
	\ifx\undefined\BibUrl\def\BibUrl#1{\url{#1}}\else\fi
	\ifx\undefined\BibAnnote\long\def\BibAnnote#1{(#1)}\else\fi
	\ifx\undefined\BibEmph\def\BibEmph#1{\emph{#1}}\else\fi
	
	
	
	\bibitem{1}
	\selectlanguageifdefined{russian}
	Умножение матриц. URL:
	http://www.algolib.narod.ru/Math/Matrix.html.
	
	\bibitem{2}
	\selectlanguageifdefined{russian}
	std::thread. URL:  https://ru.cppreference.com/w/cpp/thread/thread.
	
	\bibitem{3}
	\selectlanguageifdefined{russian}
	<ctime> (time.h). URL:  http://www.cplusplus.com/reference/ctime/.
	
	\bibitem{4}
	\selectlanguageifdefined{russian}
	Стандарт языка С++11 согласно ISO. URL:
	https://isocpp.org/std/the-standard.
	
\end{thebibliography}

%\appendix   % Тут идут приложения

\end{document}