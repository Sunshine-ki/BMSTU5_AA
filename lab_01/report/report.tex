\documentclass[12pt]{report}
\usepackage[utf8]{inputenc}
\usepackage[russian]{babel}
\usepackage{listings}

\usepackage{graphicx}
\graphicspath{{./img/}}

% Для листинга кода:
\lstset{ %
language=python,                 % выбор языка для подсветки
basicstyle=\small\sffamily, % размер и начертание шрифта для подсветки кода
numbers=left,               % где поставить нумерацию строк (слева\справа)
numberstyle=\tiny,           % размер шрифта для номеров строк
stepnumber=1,                   % размер шага между двумя номерами строк
numbersep=5pt,                % как далеко отстоят номера строк от подсвечиваемого кода
showspaces=false,            % показывать или нет пробелы специальными отступами
showstringspaces=false,      % показывать или нет пробелы в строках
showtabs=false,             % показывать или нет табуляцию в строках
frame=single,              % рисовать рамку вокруг кода
tabsize=2,                 % размер табуляции по умолчанию равен 2 пробелам
captionpos=t,              % позиция заголовка вверху [t] или внизу [b] 
breaklines=true,           % автоматически переносить строки (да\нет)
breakatwhitespace=false, % переносить строки только если есть пробел
escapeinside={\#*}{*)}   % если нужно добавить комментарии в коде
}

% Для измененных титулов глав:
\usepackage{titlesec, blindtext, color} % подключаем нужные пакеты
\definecolor{gray75}{gray}{0.75} % определяем цвет
\newcommand{\hsp}{\hspace{20pt}} % длина линии в 20pt
% titleformat определяет стиль
\titleformat{\chapter}[hang]{\Huge\bfseries}{\thechapter\hsp\textcolor{gray75}{|}\hsp}{0pt}{\Huge\bfseries}

\usepackage{pgfplots}
\usepackage{filecontents}
\usetikzlibrary{datavisualization}
\usetikzlibrary{datavisualization.formats.functions}
\begin{filecontents}{LevR.dat}
	3 0.00006
	4 0.00033
	5 0.00033
	6 0.00780
	7 0.03876
	8 0.20780
	9 1.18171
\end{filecontents}

\begin{filecontents}{LevT.dat}
	3 0.00003
	4 0.00003
	5 0.00005
	6 0.00005
	7 0.00007
	8 0.00008
	9 0.00009
\end{filecontents}

\begin{filecontents}{DamLevR.dat}
	3 0.00006
	4 0.00027
	5 0.00143
	6 0.00787
	7 0.04130
	8 0.23259
	9 1.26665
\end{filecontents}

\begin{filecontents}{DamLevT.dat}
	3 0.00003
	4 0.00003
	5 0.00005
	6 0.00006
	7 0.00007
	8 0.00013
	9 0.00012
\end{filecontents}


\begin{document}
	

%\def\chaptername{} % убирает "Глава"
\begin{titlepage}
		\noindent \begin{minipage}{0.15\textwidth}
		\includegraphics[width=\linewidth]{img/b_logo}
	\end{minipage}
	\noindent\begin{minipage}{0.9\textwidth}\centering
		\textbf{Министерство науки и высшего образования Российской Федерации}\\
		\textbf{Федеральное государственное бюджетное образовательное учреждение высшего образования}\\
		\textbf{«Московский государственный технический университет имени Н.Э.~Баумана}\\
		\textbf{(национальный исследовательский университет)»}\\
		\textbf{(МГТУ им. Н.Э.~Баумана)}
	\end{minipage}

	\vspace{3cm}
	\centering
	{\scshape\Large Лабораторная работа №1\par}
	\vspace{0.5cm}	
	{\scshape\Large По курсу: "Анализ алгоритмов"\par}
	\vspace{1.5cm}
	{\huge\bfseries Расстояние Левенштейна\par}
	\vspace{2cm}
	\Large Работу выполнила: Сукочева Алис, ИУ7-53Б\par
	\vspace{0.5cm}
	\LargeПреподаватели:  Волкова Л.Л., Строганов Ю.В.\par



	\vfill
	\large \textit {Москва, 2020} \par
\end{titlepage}
	

\tableofcontents

\newpage
\chapter*{Введение}
\addcontentsline{toc}{chapter}{Введение}

В данной лабораторной работе мы познакомимся с расстоянием Левенштейна. Данное расстояние показывает нам минимальное количество редакторских операций (вставки, замены и удаления), которые необходимы нам для перевода одной строки в другую. Это расстояние помогает определить схожесть двух строк.

Какие ошибки человек может допускать при написании какого-то текста? Например, его пальцы могут нажимать на нужные клавиши не в том порядке. С этой проблемой поможет нам справится расстояние Дамерау-Левенштейна. Данное расстояние задействует еще одну редакторскую операцию - транспозицию.

Практическое применение расстояние Левенштейна:

\begin{itemize}
	\item Сравнение введенной строки со словарными словами в поисковой системе, такой как 'yandex' или 'google'
	\item Помогает найти разницу двух ДНК. Оценка мутации.
\end{itemize}

Целью данной работы является разбор и реализация алгоритма Дамерау-Левенштейна и Левенштейна.

В рамках выполнения работы необходимо решить следующие задачи:

\begin{enumerate}
	\item Изучить алгоритмы Дамерау-Левенштейна и Левенштейна.
	\item Реализовать изученные алгоритмы, а также матричную и рекурсивную реализацию алгоритма.
	\item Подсчет времени поиска расстояния.
	\item Сравнить временные характеристики, а также затраченную память.
	\item Описать выбранную среду разработки и ЯП.
\end{enumerate}

\chapter{Аналитическая часть}

\section{Некоторые теоретические сведения}

При преобразовании одного слова в другое мы можем использовать следующие операции:

\begin{enumerate}
	\item D (\textit{от англ. delete}) - удаление.
	\item I (\textit{от англ. insert}) - вставка.
	\item R (\textit{от англ. replace}) - замена.
\end{enumerate}

Будем считать стоимость каждой вышеизложенной операции - 1.

Введем понятие совпадения - M (\textit{от англ. match}). Его стоимость будет равна 0.

\section{Расстояние Левенштейна}

Имеем две строки $S_{1}$ и $S_{2}$, длинной M и N соответственно.
Расстояние Левенштейна рассчитывается по приведенной ниже рекуррентной формуле.

!!! По формуле и ссылка на нее

%\textrm{}
% Формулы внутри текста окружаются знаками $.
\begin{displaymath}
D(S_1[1...i],S_2[1...j]) = \left\{ \begin{array}{ll}
$j, если i == 0$\\
$i, если j == 0$\\
min(\\
D(S_1[1...i],S_2[1...j-1]) + 1,\\
D(S_1[1...i-1], S_2[1...j]) + 1, & $j>0, i>0$\\
D(S_1[1...i-1], S_2[1...j-1]) + \\
\left[ 
\begin{array}{c} 
$0, если $S_1$[i] == $S_2$[j]$\\
$1, иначе$
\end{array}
\right.\\
),
\end{array} \right.
\end{displaymath}

\section{Расстояние Дамерау-Левенштейна}

Как было написано выше, в расстоянии Дамерау-Левенштейна задействует еще одну редакторскую операцию - транспозицию T (\textit{от англ. transposition}). Рекуррентная формула Дамерау-Левенштейна представлена ниже.

\begin{displaymath}
D(S_1[1...i],S_2[1...j]) = \left\{ \begin{array}{ll}
$j, если i == 0$\\
$i, если j == 0$\\
min(\\
D(S_1[1...i],S_2[1...j-1]) + 1,\\
D(S_1[1...i-1], S_2[1...j]) + 1, & \textrm{$j>0, i>0$}\\
D(S_1[1...i-1], S_2[1...j-1]) + \\
\left[ 
\begin{array}{c} 
$0, если $S_1$[i] == $S_2$[j]$\\
$1, иначе$
\end{array}
\right.\\
D(S_1[1...i-2],S_2[1...j-2]) + 1, & $i, j>1, $a_i=b_{j-1}, b_j=a_{i-1}$$\\
),
\end{array} \right.
\end{displaymath}

\section{Вывод}

Мы познакомились с основополагающими материалами, которые в дальнейшем помогут нам при реализации алгоритмов Левенштейна и Дамерау-Левенштейна.  





\chapter{Конструкторская часть}
В данном разделе мы рассмотрим схемы вышеизложенных алгоритмов.

\section{Разработка алгоритмов}

\begin{figure}[ht!]
	\centering{
		\includegraphics[width=0.6\textwidth]{img/diagramLev.png}
		\caption{Схема алгоритма Левенштейна}}
\end{figure}

\begin{figure}[ht!]
	\centering{
		\includegraphics[width=1\textwidth]{img/diagramDamLev.png}
		\caption{Схема алгоритма Дамерау-Левенштейна}}
\end{figure}

\begin{figure}[ht!]
	\centering{
		\includegraphics[width=0.8\textwidth]{img/diagramLevRec.png}
		\caption{Схема рекурсивного алгоритма Левенштейна}}
\end{figure}

!!! На рисунке таком-то мы разобрали то-то и то-то
%\section{Вывод}
%
%В данном разделе мы рассмотрели схемы алгоритмов Левенштейна и Дамерау-Левенштейна.

\chapter{Технологическая часть}
\section{Выбор яп, модули, тесты...}


%\include{research}

\chapter*{Заключение}
\addcontentsline{toc}{chapter}{Заключение}
В этой лабораторной работе мы познакомились с алгоритмами Левенштейна и Дамерау-Левенштейна. Построили схемы, соответствующие данным алгоритмам, также разобрали рекуррентные реализации. Написали полностью готовый и протестированный программный продукт, который считает дистанцию 4 способами. 
%Убедились, что алгоритм, реализованный с помощью рекурсии работает дольше, зато его преимущество в простоте написания.

\end{document}