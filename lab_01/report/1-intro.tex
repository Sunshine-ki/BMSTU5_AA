\Introduction

В данной лабораторной работе мы познакомимся с расстоянием Левенштейна. Данное расстояние показывает нам минимальное количество редакторских операций (вставки, замены и удаления), которые необходимы нам для перевода одной строки в другую. Это расстояние помогает определить схожесть двух строк.

Какие ошибки человек может допускать при написании какого-то текста? Например, его пальцы могут нажимать на нужные клавиши не в том порядке. С этой проблемой поможет нам справится расстояние Дамерау-Левенштейна. Данное расстояние задействует еще одну редакторскую операцию - транспозицию.

Практическое применение расстояние Левенштейна:

\begin{itemize}
	\item Сравнение введенной строки со словарными словами в поисковой системе, такой как 'yandex' или 'google'
	\item Помогает найти разницу двух ДНК. Оценка мутации.
\end{itemize}

Целью данной работы является разбор и реализация алгоритма Дамерау-Левенштейна и Левенштейна.

В рамках выполнения работы необходимо решить следующие задачи:

\begin{enumerate}
	\item Изучить алгоритмы Дамерау-Левенштейна и Левенштейна.
	\item Реализовать изученные алгоритмы, а также матричную и рекурсивную реализацию алгоритма.
	\item Подсчет времени поиска расстояния.
	\item Сравнить временные характеристики, а также затраченную память.
	\item Описать выбранную среду разработки и ЯП.
\end{enumerate}