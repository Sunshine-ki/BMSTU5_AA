
\chapter{ Аналитический раздел}
\label{cha:analysis}

\section{Некоторые теоретические сведения}

При преобразовании одного слова в другое мы можем использовать следующие операции:

\begin{enumerate}
	\item D (\textit{от англ. delete}) - удаление.
	\item I (\textit{от англ. insert}) - вставка.
	\item R (\textit{от англ. replace}) - замена.
\end{enumerate}

Будем считать стоимость каждой вышеизложенной операции - 1.

Введем понятие совпадения - M (\textit{от англ. match}). Его стоимость будет равна 0.

\section{Расстояние Левенштейна}

Имеем две строки $S_{1}$ и $S_{2}$, длинной M и N соответственно.
Расстояние Левенштейна рассчитывается по рекуррентной формуле (\ref{eq:ref1}).

%\textrm{}
% Формулы внутри текста окружаются знаками $.
\begin{equation}
D(S_1[1...i],S_2[1...j]) = \left\{ \begin{array}{ll}
$j, если i == 0$\\
$i, если j == 0$\\
min(\\
D(S_1[1...i],S_2[1...j-1]) + 1,\\
D(S_1[1...i-1], S_2[1...j]) + 1, & $j>0, i>0$\\
D(S_1[1...i-1], S_2[1...j-1]) + \\
\left[ 
\begin{array}{c} 
$0, если $S_1$[i] == $S_2$[j]$\\
$1, иначе$
\end{array}
\label{eq:ref1}
\right.\\
)
\end{array} \right.
\end{equation}

\section{Расстояние Дамерау-Левенштейна}

Как было написано выше, в расстоянии Дамерау-Левенштейна задействует еще одну редакторскую операцию - транспозицию T (\textit{от англ. transposition}). 
Расстояние Дамерау-Левенштейна рассчитывается по рекуррентной формуле. (\ref{eq:ref2}).

\begin{equation}
D(S_1[1...i],S_2[1...j]) = \left\{ \begin{array}{ll}
$j, если i == 0$\\
$i, если j == 0$\\
min(\\
D(S_1[1...i],S_2[1...j-1]) + 1,\\
D(S_1[1...i-1], S_2[1...j]) + 1, & \textrm{$j>0, i>0$}\\
D(S_1[1...i-1], S_2[1...j-1]) + \\
\left[ 
\begin{array}{c} 
$0, если $S_1$[i] == $S_2$[j]$\\
$1, иначе$
\end{array}
\right.\\
D(S_1[1...i-2],S_2[1...j-2]) + 1, & $i, j>1, $a_i=b_{j-1}, b_j=a_{i-1}$$\\
)
\label{eq:ref2}
\end{array} \right.
\end{equation}

\section{Вывод}

Мы познакомились с основополагающими материалами, которые в дальнейшем помогут нам при реализации алгоритмов Левенштейна и Дамерау-Левенштейна.  



