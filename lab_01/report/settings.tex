% Для листинга кода:
\lstset{ %
language=python,                 % выбор языка для подсветки
basicstyle=\small\sffamily, % размер и начертание шрифта для подсветки кода
numbers=left,               % где поставить нумерацию строк (слева\справа)
numberstyle=\tiny,           % размер шрифта для номеров строк
stepnumber=1,                   % размер шага между двумя номерами строк
numbersep=5pt,                % как далеко отстоят номера строк от подсвечиваемого кода
showspaces=false,            % показывать или нет пробелы специальными отступами
showstringspaces=false,      % показывать или нет пробелы в строках
showtabs=false,             % показывать или нет табуляцию в строках
frame=single,              % рисовать рамку вокруг кода
tabsize=2,                 % размер табуляции по умолчанию равен 2 пробелам
captionpos=t,              % позиция заголовка вверху [t] или внизу [b] 
breaklines=true,           % автоматически переносить строки (да\нет)
breakatwhitespace=false, % переносить строки только если есть пробел
escapeinside={\#*}{*)}   % если нужно добавить комментарии в коде
}

% Для измененных титулов глав:
\usepackage{titlesec, blindtext, color} % подключаем нужные пакеты
\definecolor{gray75}{gray}{0.75} % определяем цвет
\newcommand{\hsp}{\hspace{20pt}} % длина линии в 20pt
% titleformat определяет стиль
\titleformat{\chapter}[hang]{\Huge\bfseries}{\thechapter\hsp\textcolor{gray75}{|}\hsp}{0pt}{\Huge\bfseries}

\usepackage{pgfplots}
\usepackage{filecontents}
\usetikzlibrary{datavisualization}
\usetikzlibrary{datavisualization.formats.functions}
\begin{filecontents}{LevR.dat}
	3 0.00006
	4 0.00033
	5 0.00033
	6 0.00780
	7 0.03876
	8 0.20780
	9 1.18171
\end{filecontents}

\begin{filecontents}{LevT.dat}
	3 0.00003
	4 0.00003
	5 0.00005
	6 0.00005
	7 0.00007
	8 0.00008
	9 0.00009
\end{filecontents}

\begin{filecontents}{DamLevR.dat}
	3 0.00006
	4 0.00027
	5 0.00143
	6 0.00787
	7 0.04130
	8 0.23259
	9 1.26665
\end{filecontents}

\begin{filecontents}{DamLevT.dat}
	3 0.00003
	4 0.00003
	5 0.00005
	6 0.00006
	7 0.00007
	8 0.00013
	9 0.00012
\end{filecontents}
