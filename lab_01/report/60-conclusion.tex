\Conclusion % заключение к отчёту

Алгоритмы Левенштейна и ДамерауЛевенштейна являются самыми популярными алгоритмами, которые помогают найти редакторское расстояние. \\
В этой лабораторной работе мы познакомились с алгоритмами Левенштейна (Формула \ref{eq:ref1}) и Дамерау-Левенштейна (Формула \ref{eq:ref2}).
Построили схемы (Схема \ref{fg:ref1}, Схема \ref{fg:ref2}), соответствующие данным алгоритмам, также разобрали рекуррентные реализации (Схема \ref{fg:ref3}).
Написали полностью готовый и протестированный (Таблица \ref{table:ref1}) программный продукт, который считает дистанцию 4 способами.

В рамках выполнения работы мы решили следующие задачи:

\begin{enumerate}
	\item Изучили алгоритмы Дамерау-Левенштейна и Левенштейна.
	\item Реализовали изученные алгоритмы, а также матричную и рекурсивную реализации алгоритма.
	\item Проиллюстрировали алгоритмы на схемах.
	\item Описали выбранную среду разработки и ЯП.
	\item Сравнили временные характеристики, а также затраченную память.
\end{enumerate}

По окончании изучения данного материала можно смело идти и реализовывать алгоритмы нахождения редакционного расстояния!